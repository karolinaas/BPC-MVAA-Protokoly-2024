\documentclass[a4paper, czech]{article}

\usepackage[czech]{babel}
\usepackage{indentfirst}
\usepackage{graphicx}
\usepackage{float}
\usepackage[margin=1.5cm]{geometry}
\usepackage{booktabs}
\usepackage{amsmath}
\usepackage{xcolor}
\usepackage{multirow}
\usepackage{tabularray}
\usepackage{bold-extra}
\usepackage{circuitikz}

\begin{document}
\begin{table}[H]
    \centering
    \begin{tblr}{
        cell{1}{1} = {c = 2, r = 4}{c}, % Logo
        cell{1}{4} = {c = 3}{c}, % Předmět
        cell{2}{4} = {c = 3}{c}, % Jméno
        cell{3}{4} = {}{c}, % Ročník
        cell{3}{6} = {}{c}, % Studijní skupina
        cell{4}{4} = {}{c}, % Spolupracoval
        cell{4}{6} = {}{c}, % Mereno dne
        cell{5}{1} = {c = 2}{55mm}, % Kontroloval
        cell{5}{3} = {c = 2}{55mm}, % Hodnoceni
        cell{5}{5} = {c = 2}{55mm}, % Dne
        cell{6}{2} = {c = 5}{}, % Nazev ulohy
        cell{7}{1} = {}{c}, % Číslo úlohy
        cell{7}{2} = {c = 5}{c}, % Název úlohy
        vline{1,2,7} = {1.2pt},
        vline{3,5},
        hline{1,5,6,8} = {1.2pt},
        hline{2,3,4}
        }
        \includegraphics{logo_fekt.png} & & \textsuperscript{Předmět} & \large \textbf{Měření v audiotechnice} \\
             & & \textsuperscript{Jméno} & \large \textbf{Karolína Šebestová} \\
             & & \textsuperscript{Ročník} & \large \textbf{3.} & \textsuperscript{Studijní skupina} & \large \textbf{St 14:00} \\
             & & \textsuperscript{Spolupracoval} & \large \textbf{Filip Kokavec} & \textsuperscript{Měřeno dne} & \large \textbf{30.10.2024} \\
        \textsuperscript{Kontroloval} & & \textsuperscript{Hodnocení} & & \textsuperscript{Dne} \\
        \textsuperscript{Číslo úlohy} & \textsuperscript{Název úlohy} \\
        \Large \textbf{5B} & \Large \textsc{\textbf{Měřicí převodníky s operačním zesilovačem}} \\
    \end{tblr}
\end{table}

\section{Zadání}

\begin{itemize}
    \item Obeznamte se s vlastnostmi operačního zesilovače ve funkci převodníku v měřicí technice.
    \item Proměřte základní zapojení operačního zesilovače jako lineárního převodníku U/U, U/I, I/U a I/I, dále logaritmického převodníku logU/U.
    \item Určete hodnotu převodní konstanty a pracovní rozsah vstupních hodnot jednotlivých převodníků.
    \item Určete vstupní odpor převodníků.
\end{itemize}

\section{Teoretický úvod}

Zesilovač je zařízení, které je schopno transformací elektrické energie z vnějšího zdroje měnit výstupní parametry.
Ideálním zesilovačem nazýváme takový zesilovač, který má nekonečné zesílení a té nekonečný vstupní odpor.
Volbou použité zpětné vazby zesilovače je možno vytvořit čtyři základní lineární převodníky.
Častým je také využitá nelineárních převodníků, které vykazují obvyklé matematické funkce.
Známe zapojení zesilovačů jak invertující tak neinvertující. Neinvertující zapojení mají polaritu výstupního signálu shodnou se vstupním signálem.

\begin{itemize}
    \item \textbf{Převodník U/U} - vstupní napětí zesilovače se ve zpětnovazebném zapojení blíží nule. Převodní konstanta A je určena vztahem:
    \begin{equation*}
        A = \frac{U_2}{U_1}
    \end{equation*}

    \item \textbf{Převodník U/I} - nevýhoda tohoto převodníku je, že výstup není spojen se společným potenciálem (je plovoucí). Převodní konstanta S je definována vztahem:
    \begin{equation*}
        S = \frac{I_2}{U_1}
    \end{equation*}

    \item \textbf{Převodník I/U} - převodní konstanta W je určena vztahem:
    \begin{equation*}
        W = \frac{U_2}{I_1}
    \end{equation*}

    \item \textbf{Převodník I/I} - převodní konstanta (v tomto případě proudové zesílení) je definována vztahem:
    \begin{equation*}
        B = \frac{I_2}{I_1}
    \end{equation*}

    \item \textbf{Převodník logU/U} - tento převodník využívá exponenciální závislosti mezi proudem kolektoru a proudem v bázi tranzistoru. Jeho nevýhodou je, že tranzistor je teplotně závislý a pro skutečné využití v praxi je potřeba zkonstruovat také teplotní kompenzaci. Parametr $K_1$ udává směrnici logaritmické závislosti a $K_2$ vertikální napěťový posun.
    \begin{equation*}
        U_2 = K_1 \cdot log (U_1) + K_2
    \end{equation*}
\end{itemize}

\begin{figure}[H]
    \centering
    \begin{circuitikz}
        \draw (0,0) to[european resistor, l=$R_0$] (2,0)
        node[ocirc, scale=2]{} to[rmeter, t=mA$_1$, f=$I_1$] (4,0)
        to[] (5,0) node[circ]{}
        to[rmeter, t=V$_1$] (5,-2) node[circ]{}
        to[] (2,-2) node[ocirc, scale=2]{}
        to[] (0,-2)
        to[dcvsource, name=zdroj] (0,0);
        \ctikztunablearrow{1}{1.25}{-45}{zdroj}

        \draw (5,0) to[] (6,0) node[ocirc, scale=2]{} to[] (7,0);
        \draw (5,-2) to[] (6,-2) node[ocirc, scale=2]{} to[] (7,-2);

        \draw (7,0.5) rectangle (10,-2.5) node[pos=0.5, align=center] {\Large Měřený\\\Large převodník};

        \draw (10,0) to[] (11,0) node[ocirc, scale=2]{};
        \draw (10,-2) to[] (11,-2) node[ocirc, scale=2]{};

        \draw[{Straight Barb[scale=2]}-]  (11.5,0) to[] (12.5,0);
        \draw (12.5,0) to[rmeter, t=V$_2$] (12.5,-2);
        \draw[{Straight Barb[scale=2]}-]  (11.5,-2) to[] (12.5,-2);

        \draw[{Straight Barb[scale=2]}-]  (13,0) to[] (14,0);
        \draw (14,0) to[rmeter, t=mA$_2$] (14,-2);
        \draw[{Straight Barb[scale=2]}-]  (13,-2) to[] (14,-2);
        \draw (13.5,0.25) node[flowarrow]{$I_2$};

        \draw[-{Straight Barb[angle'=60, scale=2]}]  (6,-0.25) to[] (6,-1.75);
        \draw (6.3,-1) node[]{$U_1$};

        \draw[-{Straight Barb[angle'=60, scale=2]}]  (11,-0.25) to[] (11,-1.75);
        \draw (11.3,-1) node[]{$U_2$};

        \draw (-1,-1) node[align=center]{Zdroj\\0-15\,V};
        \draw (12,-2.4) node[]{buď};
        \draw (13.5,-2.4) node[]{nebo};

    \end{circuitikz}
    \caption{Zapojení pro měření převodních charakteristik zesilovačů}
\end{figure}

\section{Výsledky měření}

\subsection{Tabulky}

\begin{minipage}{0.48\textwidth}
    \begin{table}[H]
        \catcode`\-=12
        \centering
        \caption{Převodník U/U}
        \begin{tabular}{ccccc}
            \toprule
            $U_1$  & $I_1$  & $U_2$    & $R_{vst}$  & $A$     \\
            \cmidrule(rl){1-1}
            \cmidrule(rl){2-2}
            \cmidrule(rl){3-3}
            \cmidrule(rl){4-4}
            \cmidrule(rl){5-5}
            V   & $\mu$A  & V     & M$\Omega$    & -     \\
            \midrule
            0   & 0   & 0     & -     & -     \\
            0,5 & 0   & 1,255 & -     & 2,510 \\
            1,0 & 0,1 & 2,499 & 10,00 & 2,499 \\
            1,5 & 0,1 & 3,767 & 15,00 & 2,511 \\
            2,0 & 0,2 & 4,980 & 10,00 & 2,490 \\
            2,5 & 0,2 & 6,240 & 12,50 & 2,496 \\
            3,0 & 0,3 & 7,480 & 10,00 & 2,493 \\
            3,5 & 0,3 & 8,700 & 11,67 & 2,486 \\
            4,0 & 0,4 & 10,01 & 10,00 & 2,503 \\
            4,5 & 0,5 & 11,24 & 9,000 & 2,498 \\
            5,0 & 0,6 & 12,52 & 8,333 & 2,504 \\
            5,5 & 0,6 & 13,75 & 9,167 & 2,500 \\
            6,0 & 0,7 & 15,00 & 8,571 & 2,500 \\
            6,5 & 0,8 & 15,07 & 8,125 & 2,318 \\
            7,0 & 0,8 & 15,07 & 8,750 & 2,153 \\
            7,5 & 0,9 & 15,07 & 8,333 & 2,009 \\
            \bottomrule
            \multicolumn{5}{l}{Pracovní oblast: $U_1 = \{0$ až $6\}$ V} \\
            \multicolumn{5}{l}{Převodní konst.: $A = 2,431$} \\
            \multicolumn{5}{l}{$R_{vst} = 10$ M$\Omega$} \\
        \end{tabular}
    \end{table}
\end{minipage}
\begin{minipage}{0.48\textwidth}
    \begin{table}[H]
        \catcode`\-=12
        \centering
        \caption{Převodník U/I}
        \begin{tabular}{ccccc}
            \toprule
            $U_1$  & $I_1$  & $I_2$    & $R_{vst}$  & $S$     \\
            \cmidrule(rl){1-1}
            \cmidrule(rl){2-2}
            \cmidrule(rl){3-3}
            \cmidrule(rl){4-4}
            \cmidrule(rl){5-5}
            V    & mA    & mA   & k$\Omega$    & $\mu$S         \\
            \midrule
            0    & 0     & 0    & -     & -         \\
            1    & 7,500 & 0,09 & 133,3 & 90,0 \\
            2    & 15,00 & 0,19 & 133,3 & 95,0 \\
            3    & 22,50 & 0,29 & 133,3 & 96,6 \\
            4    & 30,10 & 0,39 & 132,9 & 97,5 \\
            5    & 37,50 & 0,49 & 133,3 & 98,0 \\
            6    & 45,00 & 0,58 & 133,3 & 96,6 \\
            7    & 52,50 & 0,68 & 133,3 & 97,1 \\
            8    & 60,00 & 0,78 & 133,3 & 97,5 \\
            9    & 67,50 & 0,88 & 133,3 & 97,7 \\
            10   & 75,00 & 0,98 & 133,3 & 98,0 \\
            11   & 82,50 & 1,07 & 133,3 & 97,2 \\
            12   & 90,00 & 1,17 & 133,3 & 97,5 \\
            13   & 97,50 & 1,27 & 133,3 & 97,6 \\
            14   & 105,0 & 1,37 & 133,3 & 97,8 \\
            14,2 & 106,5 & 1,39 & 133,3 & 97,8 \\
            14,4 & 108,0 & 1,40 & 133,3 & 97,2 \\
            14,6 & 109,4 & 1,43 & 133,5 & 97,9 \\
            14,8 & 111,0 & 1,44 & 133,3 & 97,3 \\
            15,0 & 112,5 & 1,47 & 133,3 & 98,0 \\
            \bottomrule
            \multicolumn{5}{l}{Pracovní oblast: $U_1 = \{0$ až $15\}$ V} \\
            \multicolumn{5}{l}{Převodní konst.: $S = 97\ \mu$S} \\
            \multicolumn{5}{l}{$R_{vst} = 133,3$ k$\Omega$} \\
        \end{tabular}
    \end{table}
\end{minipage}

\begin{minipage}{0.48\textwidth}
    \begin{table}[H]
        \catcode`\-=12
        \centering
        \caption{Převodník I/U}
        \begin{tabular}{ccccc}
            \toprule
            $I_1$   & $U_1$    & $U_2$     & $R_{vst}$  & $W$      \\
            \cmidrule(rl){1-1}
            \cmidrule(rl){2-2}
            \cmidrule(rl){3-3}
            \cmidrule(rl){4-4}
            \cmidrule(rl){5-5}
            mA   & mV    & V      & $\Omega$     & k$\Omega$     \\
            \midrule
            0    & 0     & 0      & -     & -      \\
            0,1  & 18,80 & -1,149 & 188,0 & -11,49 \\
            0,2  & 37,40 & -2,295 & 187,0 & -11,48 \\
            0,3  & 54,90 & -3,360 & 183,0 & -11,20 \\
            0,4  & 72,60 & -4,420 & 181,5 & -11,05 \\
            0,5  & 91,60 & -5,580 & 183,2 & -11,16 \\
            0,6  & 108,8 & -6,620 & 181,3 & -11,03 \\
            0,7  & 127,0 & -7,730 & 181,4 & -11,04 \\
            0,8  & 144,6 & -8,800 & 180,8 & -11,00 \\
            0,9  & 162,2 & -9,880 & 180,2 & -10,98 \\
            1,0  & 181,9 & -11,08 & 181,9 & -11,08 \\
            1,1  & 199,1 & -12,12 & 181,0 & -11,02 \\
            1,2  & 434,0 & -12,98 & 361,7 & -10,82 \\
            1,25 & 675,0 & -12,99 & 540,0 & -10,39 \\
            1,30 & 719,0 & -12,99 & 553,1 & -9,992 \\
            1,35 & 751,0 & -12,99 & 556,3 & -9,622 \\
            1,40 & 776,0 & -12,99 & 554,3 & -9,279 \\
            \bottomrule
            \multicolumn{5}{l}{Pracovní oblast: $I_1 = \{0$ až $1,2\}$ mA} \\
            \multicolumn{5}{l}{Převodní konst.: $S = -10,79$  k$\Omega$} \\
            \multicolumn{5}{l}{$R_{vst} = 285,9$ $\Omega$} \\
        \end{tabular}
    \end{table}
\end{minipage}
\begin{minipage}{0.48\textwidth}
    \begin{table}[H]
        \catcode`\-=12
        \centering
        \caption{Převodník I/I}
        \begin{tabular}{ccccc}
            \toprule
            $I_1$   & $U_1$   & $I_2$    & $R_{vst}$  & $B$      \\
            \cmidrule(rl){1-1}
            \cmidrule(rl){2-2}
            \cmidrule(rl){3-3}
            \cmidrule(rl){4-4}
            \cmidrule(rl){5-5}
            mA   & mV   & mA    & $\Omega$     & -      \\
            \midrule
            0    & 0    & 0     & -     & -      \\
            0,25 & 5,00 & -0,50 & 20,00 & -2,000 \\
            0,50 & 10,5 & -0,99 & 21,00 & -1,980 \\
            0,75 & 16,0 & -1,49 & 21,33 & -1,987 \\
            1,00 & 21,6 & -1,99 & 21,60 & -1,990 \\
            1,25 & 26,8 & -2,50 & 21,44 & -2,000 \\
            1,50 & 32,5 & -2,99 & 21,67 & -1,993 \\
            1,75 & 37,9 & -3,49 & 21,66 & -1,994 \\
            2,00 & 43,4 & -3,99 & 21,70 & -1,995 \\
            2,25 & 48,7 & -4,49 & 21,64 & -1,996 \\
            2,50 & 54,6 & -5,00 & 21,84 & -2,000 \\
            2,75 & 59,9 & -5,50 & 21,78 & -2,000 \\
            3,00 & 65,8 & -6,00 & 21,93 & -2,000 \\
            3,25 & 431  & -6,39 & 132,6 & -1,966 \\
            3,50 & 609  & -6,43 & 174,0 & -1,837 \\
            3,75 & 651  & -6,44 & 173,6 & -1,717 \\
            4,00 & 676  & -6,44 & 169,0 & -1,610 \\
            \bottomrule
            \multicolumn{5}{l}{Pracovní oblast: $I_1 = \{0$ až $3,25\}$ mA} \\
            \multicolumn{5}{l}{Převodní konst.: $B = -1,942$} \\
            \multicolumn{5}{l}{$R_{vst} = 56,68$ $\Omega$} \\
        \end{tabular}
    \end{table}
\end{minipage}

\begin{table}[H]
    \catcode`\-=12
    \centering
    \caption{Převodník logU/U}
    \begin{tabular}{cccc}
        \toprule
        $U_1$  & $I_1$  & $U_2$    & $R_{vst}$ \\
        \cmidrule(rl){1-1}
        \cmidrule(rl){2-2}
        \cmidrule(rl){3-3}
        \cmidrule(rl){4-4}
        V   & mA  & V     & k$\Omega$ \\
        \midrule
        0,1 & 0,6   & -0,481 & 166,7 \\
        0,3 & 2,0   & -0,509 & 150,0 \\
        1   & 6,9   & -0,541 & 144,9 \\
        3   & 20,7  & -0,570 & 144,9 \\
        6   & 41,5  & -0,588 & 144,6 \\
        10  & 69,0  & -0,602 & 144,9 \\
        15  & 103,7 & -0,612 & 144,6 \\
        \bottomrule
        \multicolumn{4}{l}{Pracovní oblast: $U_1 = \{0$ až $15\}$ V} \\
        \multicolumn{4}{l}{Převodní konst.: $K_1 = -0,026$ V/dek} \\
        \multicolumn{4}{l}{Převodní konst.: $K_2 = -0,5411$ V} \\
        \multicolumn{4}{l}{$R_{vst} = 148,7$ $\Omega$} \\
    \end{tabular}
\end{table}

\subsection{Grafy}

\begin{figure}[H]
    \centering
    \includegraphics[width=0.9\textwidth]{grafy/graf_prevodnik_UU.pdf}
    \caption{Převodník U/U - Závislost výstupního napětí $U_2$ na vstupním napětí $U_1$}
\end{figure}

\begin{figure}[H]
    \centering
    \includegraphics[width=0.9\textwidth]{grafy/graf_prevodnik_UI.pdf}
    \caption{Převodník U/I - Závislost výstupního proudu $I_2$ na vstupním napětí $U_1$}
\end{figure}

\begin{figure}[H]
    \centering
    \includegraphics[width=0.9\textwidth]{grafy/graf_prevodnik_IU.pdf}
    \caption{Převodník I/U - Závislost výstupního napětí $U_2$ na vstupním proudu $I_1$}
\end{figure}

\begin{figure}[H]
    \centering
    \includegraphics[width=0.9\textwidth]{grafy/graf_prevodnik_II.pdf}
    \caption{Převodník I/I - Závislost výstupního proudu $I_2$ na vstupním proudu $I_1$}
\end{figure}

\begin{figure}[H]
    \centering
    \includegraphics[width=0.9\textwidth]{grafy/graf_prevodnik_logUU.pdf}
    \caption{Převodník logU/U - Závislost výstupního napětí $U_2$ na vstupním napětí $U_1$}
\end{figure}

\subsection{Příklady výpočtu}

\begin{enumerate}
    \item Vstupní odpor převodníku
    \begin{multline*}
        R_{vst} = \textcolor{teal}{\frac{U_1}{I_1}} = \frac{1 \, V}{0,1 \cdot 10^{-6} \, A} = 10 \cdot 10^{6} \, \Omega = \underline{\underline{10 \, M \Omega}} \hfill
    \end{multline*}

    \item Převodní konstanta $A$
    \begin{multline*}
        A = \textcolor{teal}{\frac{U_2}{U_1}} = \frac{1,255 \, V}{0,5 \, V} = \underline{\underline{2,51}} \hfill
    \end{multline*}

    \item Převodní konstanta $S$
    \begin{multline*}
        S = \textcolor{teal}{\frac{I_2}{U_1}} = \frac{0,19 \cdot 10^{-3} \, A}{2 \, V} = 95 \cdot 10^{-6} \, S = \underline{\underline{95 \, \mu S}} \hfill
    \end{multline*}

    \item Převodní konstanta $W$
    \begin{multline*}
        W = \textcolor{teal}{\frac{U_2}{I_1}} = \frac{-2,295 \, V}{0,2 \cdot 10^{-3} \, A} = -11,475 \cdot 10^{3} \, \Omega = \underline{\underline{-11,48 \, k \Omega}} \hfill
    \end{multline*}

    \item Převodní konstanta $B$
    \begin{multline*}
        B = \textcolor{teal}{\frac{I_2}{I_1}} = \frac{-0,99 \cdot 10^{-3} \, A}{0,5 \cdot 10^{-3} \, A} = \underline{\underline{-1,98}} \hfill
    \end{multline*}

    \item Konstanty $K_1$ a $K_2$ - odečteno z aproximace křivky
    \begin{multline*}
        \textcolor{teal}{U_2 = K_1 \cdot log \left(U_1\right) + K_2} = -0,026 \cdot ln \left(U_1\right) - 0,5411 \hfill \\
        \ \ \ K_1 = \underline{\underline{-0,026}} \hfill \\
        \ \ \ K_2 = \underline{\underline{-0,5411}} \hfill
    \end{multline*}
\end{enumerate}

\section{Seznam použitých přístrojů}

\begin{itemize}
    \item Ruční číslicový multimetr Finest 703 TRUE RMS, v.č. 180601927
    \item Ruční číslicový multimetr Finest 703 TRUE RMS, v.č. 180601950
    \item Ruční číslicový multimetr Finest 703 TRUE RMS, v.č. 180601951
\end{itemize}

\section{Závěr}

Úkolem tohoto měření bylo ověřit několik základních zapojení operačního zesilovače. Dále také určit hodnoty převodních konstant, pracovní rozsahy a také vstupní odpory měřených převodníků.

Námi určená pracovní oblast převodníku U/U je 0-6V. Převodní konstanta je rovna hodnotě 2,431 a vstupní odpor je roven hodnotě 10\,M$\Omega$. Tento převodník je neinvertující.

Námi určená pracovní oblast převodníku U/I je 0-15V. Převodní konstanta je rovna hodnotě 97\,$\mu$S a vstupní odpor je roven hodnotě 133,3\,k$\Omega$. Tento převodník je neinvertující.

Námi určená pracovní oblast převodníku I/U je 0-1,2mA. Převodní konstanta je rovna hodnotě -10,79\,k$\Omega$ a vstupní odpor je roven hodnotě 285,9\,$\Omega$. Tento převodník je invertující.

Námi určená pracovní oblast převodníku I/I je 0-3,25mA. Převodní konstanta je rovna hodnotě -1,942 a vstupní odpor je roven hodnotě 56,68\,$\Omega$. Tento převodník je invertující.

Námi určená pracovní oblast převodníku logU/U je 0-15V. Konstanta $K_1$ je rovna hodnotě -0,026\,V/dek a konstanta $K_2$ je rovna -0,5411 V. Vstupní odpor je roven hodnotě 148,7\,$\Omega$. Tento převodník je invertující.

Námi naměřené hodnoty odpovídají našim teoretickým předpokladům. Lze tedy usoudit, že nedošlo k žádným větším systematickým chybám měření.


\end{document}