\documentclass[a4paper, czech]{article}

\usepackage[czech]{babel}
\usepackage{indentfirst}
\usepackage{graphicx}
\usepackage{float}
\usepackage[margin=1.5cm]{geometry}
\usepackage{booktabs}
\usepackage{amsmath}
\usepackage{xcolor}
\usepackage{multirow}
\usepackage{tabularray}
\usepackage{bold-extra}

\begin{document}
\begin{table}[H]
    \centering
    \begin{tblr}{
        cell{1}{1} = {c = 2, r = 4}{c}, % Logo
        cell{1}{4} = {c = 3}{c}, % Předmět
        cell{2}{4} = {c = 3}{c}, % Jméno
        cell{3}{4} = {}{c}, % Ročník
        cell{3}{6} = {}{c}, % Studijní skupina
        cell{4}{4} = {}{c}, % Spolupracoval
        cell{4}{6} = {}{c}, % Mereno dne
        cell{5}{1} = {c = 2}{55mm}, % Kontroloval
        cell{5}{3} = {c = 2}{55mm}, % Hodnoceni
        cell{5}{5} = {c = 2}{55mm}, % Dne
        cell{6}{2} = {c = 5}{}, % Nazev ulohy
        cell{7}{1} = {}{c}, % Číslo úlohy
        cell{7}{2} = {c = 5}{c}, % Název úlohy
        vline{1,2,7} = {1.2pt},
        vline{3,5},
        hline{1,5,6,8} = {1.2pt},
        hline{2,3,4}
        }
        \includegraphics{logo_fekt.png} & & \textsuperscript{Předmět} & \large \textbf{Měření v audiotechnice} \\
             & & \textsuperscript{Jméno} & \large \textbf{Karolína Šebestová} \\
             & & \textsuperscript{Ročník} & \large \textbf{3.} & \textsuperscript{Studijní skupina} & \large \textbf{St 14:00} \\
             & & \textsuperscript{Spolupracoval} & \large \textbf{Filip Kokavec} & \textsuperscript{Měřeno dne} & \large \textbf{30.10.2024} \\
        \textsuperscript{Kontroloval} & & \textsuperscript{Hodnocení} & & \textsuperscript{Dne} \\
        \textsuperscript{Číslo úlohy} & \textsuperscript{Název úlohy} \\
        \Large \textbf{5B} & \Large \textsc{\textbf{Měřicí převodníky s operačním zesilovačem}} \\
    \end{tblr}
\end{table}

\section{Zadání}

\begin{itemize}
    \item Obeznamte se s vlastnostmi operačního zesilovače ve funkci převodníku v měřicí technice.
    \item Proměřte základní zapojení operačního zesilovače jako lineárního převodníku U/U, U/I, I/U a I/I, dále logaritmického převodníku logU/U.
    \item Určete hodnotu převodní konstanty a pracovní rozsah vstupních hodnot jednotlivých převodníků.
    \item Určete vstupní odpor převodníků.
\end{itemize}

\section{Teoretický úvod}

\section{Výsledky měření}

\subsection{Tabulky}

\subsection{Grafy}

\begin{figure}[H]
    \centering
    \includegraphics[width=0.9\textwidth]{grafy/graf_prevodnik_UU.pdf}
    \caption{Převodník U/U - Závislost výstupního napětí $U_2$ na vstupním napětí $U_1$}
\end{figure}

\begin{figure}[H]
    \centering
    \includegraphics[width=0.9\textwidth]{grafy/graf_prevodnik_UI.pdf}
    \caption{Převodník U/I - Závislost výstupního proudu $I_2$ na vstupním napětí $U_1$}
\end{figure}

\begin{figure}[H]
    \centering
    \includegraphics[width=0.9\textwidth]{grafy/graf_prevodnik_IU.pdf}
    \caption{Převodník I/U - Závislost výstupního napětí $U_2$ na vstupním proudu $I_1$}
\end{figure}

\begin{figure}[H]
    \centering
    \includegraphics[width=0.9\textwidth]{grafy/graf_prevodnik_II.pdf}
    \caption{Převodník I/I - Závislost výstupního proudu $I_2$ na vstupním proudu $I_1$}
\end{figure}

\begin{figure}[H]
    \centering
    \includegraphics[width=0.9\textwidth]{grafy/graf_prevodnik_logUU.pdf}
    \caption{Převodník log U/U - Závislost výstupního napětí $U_2$ na vstupním napětí $U_1$}
\end{figure}

\subsection{Příklady výpočtu}

\section{Seznam použitých přístrojů}

\section{Závěr}

\end{document}