\documentclass[a4paper, czech]{article}

\title{Úloha č.1: Stanovení nejistot při přímém měření výkonu}
\author{Karolína Andrea Šebestová \and Filip Kokavec}
\date{Datum měření: 2.10.2024}

\usepackage[czech]{babel}
\usepackage{indentfirst}
\usepackage{graphicx}
\usepackage{float}
\usepackage[margin=1.5cm]{geometry}
\usepackage{booktabs}
\usepackage{amsmath}

\begin{document}

\maketitle

\section{Zadání}

\begin{itemize}
    \item Seznamte se s měřením stejnosměrného proudu a napětí pomocí ČMP a s určením nejistoty nepřímého měření výkonu.
    \item Změřte výkon stejnosměrného proudu nepřímou metodou.
    \item Naměřené hodnoty zpracujte z hlediska odchylky metody a nejistot měření.
\end{itemize}

\section{Teoretický úvod}

Výkon stejnosměrného obvodu lze jednoduše měřit nepřímou metodou, kdy nejprve měříme elektrické napětí voltmetrem a elektrický proud ampérmetrem.
Následně je možné z naměřených hodnot napětí a proudu vypočíst výkon stejnosměrného proudu následujícím jednoduchým součinovým vztahem:

$$P_Z = U_Z \cdot I_Z$$

Pokud chceme měřit napětí i proud v obvodu současně, tak můžeme využít dvou možných zapojení.
Bohužel ani v jednom případě není možné měřit skutečné hodnoty proudu a napětí v obvodu.
V obou případech je výkon vypočtený z naměřených hodnot měřících přístrojů ovlivněn odchylkou použité metody.

Známe dvě různé obvodové uspořádání pro měření výkonu stejnosměrného obvodu:

\begin{enumerate}
    \item \textbf{VA uspořádání}

    Voltmetr v zapojení předchází ampérmetr, t.j. je zapojen paralelně k zátěži s ampérmetrem, který je se zátěží v sérii.

    Voltmetr měří součet úbytků napětí vznikajících na zátěži i ampérmetru.
    Ampérmetr měří pouize proud protékající zátěží obvodu.

    U tohoto uspořádání dochází k systematické odchylce, která je způsobena parazitními ztrátami na vnitřním odporu ampérmetru.

    \item \textbf{AV uspořádání}

    Ampérmetr předchází voltmetr, t.j. je zapojen sériově se zátěží s voltmetrem, který je k zátěži připojen paralelně.

    Ampérmetr měří součet proudů protékajících voltmetrem i zátěží obvodu.
    Voltmetr měří pouze úbytek napětí vznikající na zátěži obvodu.

    U tohoto uspořádání dochází k systematické odchylce, která je způsobena parazitními ztrátami na vnitřním odporu voltmetru.
\end{enumerate}

Systematické chyby metody způsobené měřením je potřeba zohlednit a řádně je kompenzovat.
Zároveň je také potřeba určit nejistotu měření.
Nejistotu měření určíme pomocí zákona o šíření nejistot, kdy při zanedbání nejistoty vnitřního odporu prvního přístroje v obvodu je standardní nejistota výsledku měření kombinací celkové nejistoty výsledku nepřímého měření výkonu. 

\subsection{Příklady výpočtu}

\section{Seznam použitých přístrojů}

\section{Závěr}

\end{document}