\documentclass[a4paper, czech]{article}

\usepackage[czech]{babel}
\usepackage{indentfirst}
\usepackage{graphicx}
\usepackage{float}
\usepackage[margin=1.5cm]{geometry}
\usepackage{booktabs}
\usepackage{amsmath}
\usepackage{xcolor}
\usepackage{multirow}
\usepackage{tabularray}
\usepackage{bold-extra}
\usepackage{circuitikz}

\begin{document}
\begin{table}[H]
    \centering
    \begin{tblr}{
        cell{1}{1} = {c = 2, r = 4}{c}, % Logo
        cell{1}{4} = {c = 3}{c}, % Předmět
        cell{2}{4} = {c = 3}{c}, % Jméno
        cell{3}{4} = {}{c}, % Ročník
        cell{3}{6} = {}{c}, % Studijní skupina
        cell{4}{4} = {}{c}, % Spolupracoval
        cell{4}{6} = {}{c}, % Mereno dne
        cell{5}{1} = {c = 2}{55mm}, % Kontroloval
        cell{5}{3} = {c = 2}{55mm}, % Hodnoceni
        cell{5}{5} = {c = 2}{55mm}, % Dne
        cell{6}{2} = {c = 5}{}, % Nazev ulohy
        cell{7}{1} = {}{c}, % Číslo úlohy
        cell{7}{2} = {c = 5}{c}, % Název úlohy
        vline{1,2,7} = {1.2pt},
        vline{3,5},
        hline{1,5,6,8} = {1.2pt},
        hline{2,3,4}
        }
        \includegraphics{logo_fekt.png} & & \textsuperscript{Předmět} & \large \textbf{Měření v audiotechnice} \\
             & & \textsuperscript{Jméno} & \large \textbf{Karolína Šebestová} \\
             & & \textsuperscript{Ročník} & \large \textbf{3.} & \textsuperscript{Studijní skupina} & \large \textbf{St 14:00} \\
             & & \textsuperscript{Spolupracoval} & \large \textbf{Filip Kokavec} & \textsuperscript{Měřeno dne} & \large \textbf{6.11.2024} \\
        \textsuperscript{Kontroloval} & & \textsuperscript{Hodnocení} & & \textsuperscript{Dne} \\
        \textsuperscript{Číslo úlohy} & \textsuperscript{Název úlohy} \\
        \Large \textbf{6B} & \Large \textsc{\textbf{Princip A/Č převodníku s dvojitou integrací}} \\
    \end{tblr}
\end{table}

\section{Zadání}

\begin{itemize}
    \item Obeznamte se s principem předloženého převodníku s dvojitou integrací. Proměřte jeho převodní charakteristiku, zakreslete napětí na výstupu integrátoru, ověřte jeho potlačení střídavého rušivého signálu.
    \item Graficky znázorněte naměřené převodní charakteristiky převodníku s dvojitou integrací.
\end{itemize}

\section{Teoretický úvod}

\section{Výsledky měření}

\subsection{Tabulky}

\begin{table}[H]
    \catcode`\-=12
    \centering
    \caption{Převodník s dvojitou integrací}
    \begin{tabular}{cc|ccccccccc}
        \toprule
        $U_X$  & [V]  & 0,1   & 0,2   & 0,3   & 0,4   & 0,5   & 0,6   & 0,7   & 0,8   & 0,9   \\
        \cmidrule(rl){1-11}
        $N$   & [-]  & 1000  & 2003  & 3005  & 4007  & 5008  & 6010  & 7012  & 8013  & 9015  \\
        $N_{id}$ & [-]  & 1000  & 2000  & 3000  & 4000  & 5000  & 6000  & 7000  & 8000  & 9000  \\
        $\delta_N$  & [\%] & 0,000 & 0,150 & 0,167 & 0,175 & 0,160 & 0,167 & 0,171 & 0,163 & 0,167 \\
        \bottomrule
    \end{tabular}
\end{table}

\begin{table}[H]
    \catcode`\-=12
    \centering
    \caption{Převodník s dvojitou integrací}
    \begin{tabular}{cccc}
        \toprule
        $U_X$  & $T_1$  & $T_2$ & $N_{id}$  \\
        \cmidrule(rl){1-1}
        \cmidrule(rl){2-2}
        \cmidrule(rl){3-3}
        \cmidrule(rl){4-4}
        V   & ms  & ms & -    \\
        \midrule
        0,3 & 100 & 30 & 3000 \\
        0,6 & 100 & 61 & 6100 \\
        0,9 & 100 & 91 & 9100 \\
        \bottomrule
    \end{tabular}
\end{table}

\subsection{Grafy}

\subsection{Příklady výpočtu}

\section{Seznam použitých přístrojů}

\section{Závěr}

\end{document}