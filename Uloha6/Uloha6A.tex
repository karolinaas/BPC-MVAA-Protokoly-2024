\documentclass[a4paper, czech]{article}

\usepackage[czech]{babel}
\usepackage{indentfirst}
\usepackage{graphicx}
\usepackage{float}
\usepackage[margin=1.5cm]{geometry}
\usepackage{booktabs}
\usepackage{amsmath}
\usepackage{xcolor}
\usepackage{multirow}
\usepackage{tabularray}
\usepackage{bold-extra}
\usepackage{circuitikz}

\begin{document}
\begin{table}[H]
    \centering
    \begin{tblr}{
        cell{1}{1} = {c = 2, r = 4}{c}, % Logo
        cell{1}{4} = {c = 3}{c}, % Předmět
        cell{2}{4} = {c = 3}{c}, % Jméno
        cell{3}{4} = {}{c}, % Ročník
        cell{3}{6} = {}{c}, % Studijní skupina
        cell{4}{4} = {}{c}, % Spolupracoval
        cell{4}{6} = {}{c}, % Mereno dne
        cell{5}{1} = {c = 2}{55mm}, % Kontroloval
        cell{5}{3} = {c = 2}{55mm}, % Hodnoceni
        cell{5}{5} = {c = 2}{55mm}, % Dne
        cell{6}{2} = {c = 5}{}, % Nazev ulohy
        cell{7}{1} = {}{c}, % Číslo úlohy
        cell{7}{2} = {c = 5}{c}, % Název úlohy
        vline{1,2,7} = {1.2pt},
        vline{3,5},
        hline{1,5,6,8} = {1.2pt},
        hline{2,3,4}
        }
        \includegraphics{logo_fekt.png} & & \textsuperscript{Předmět} & \large \textbf{Měření v audiotechnice} \\
             & & \textsuperscript{Jméno} & \large \textbf{Karolína Šebestová} \\
             & & \textsuperscript{Ročník} & \large \textbf{3.} & \textsuperscript{Studijní skupina} & \large \textbf{St 14:00} \\
             & & \textsuperscript{Spolupracoval} & \large \textbf{Filip Kokavec} & \textsuperscript{Měřeno dne} & \large \textbf{6.11.2024} \\
        \textsuperscript{Kontroloval} & & \textsuperscript{Hodnocení} & & \textsuperscript{Dne} \\
        \textsuperscript{Číslo úlohy} & \textsuperscript{Název úlohy} \\
        \Large \textbf{6A} & \Large \textsc{\textbf{Princip vzorkování a Č/A převodníku}} \\
    \end{tblr}
\end{table}

\section{Zadání}

\begin{itemize}
    \item Experimentálně ověřte funkci vzorkovacího zesilovače:
    \begin{itemize}
        \item znázorněte na osciloskopu průběh výstupního napětí vzorkovače při vzorkování harmonického napětí,
        \item ověřte možnost určení efektivní hodnoty napětí z navzorkovaných hodnot,
        \item v režimu ruční vzorkování sledujte na číslicovém voltmetru změnu hodnoty výstupního napětí v závislosti na čase (chyba pamatování),
        \item na osciloskopu pozorujte průběh napětí na výstupu vzorkovače během tzv. doby upnutí.
    \end{itemize}
    \item U předloženého 8bitového Č/A převodníku experimentálně proměřte převodní charakteristiku. Určete chybu nuly, chybu zesílení a maximální chybu linearity.
    \item Graficky znázorněte převodní charakteristiku Č/A převodníku.
\end{itemize}

\section{Teoretický úvod}

\section{Výsledky měření}

\subsection{Tabulky}

\begin{table}[H]
    \catcode`\-=12
    \centering
    \caption{Měření napětí vzorkovačem}
    \begin{tabular}{ll|cccccccc}
        \toprule
        $k$   & [-]  & 0      & 1      & 2      & 3      & 4      & 5      & 6      & 7     \\
        \cmidrule(rl){1-10}
        $U_k$  &  [V]  & -1,210 & -1,950 & -2,380 & -2,450 & -2,160 & -1,530 & -0,660 & 0,288 \\
        $U_k^2$ & [V$^2$] & 1,464  & 3,803  & 5,664  & 6,003  & 4,666  & 2,341  & 0,436  & 0,083 \\
        \midrule
        \midrule
        $k$   & [-]  &8     & 9     & 10    & 11    & 12    & 13    & 14    & 15     \\
        \cmidrule(rl){1-10}
        $U_k$  & [V]  &1,200 & 1,930 & 2,370 & 2,450 & 2,140 & 1,520 & 0,660 & -0,290 \\
        $U_k^2$ & [V$^2$] &1,440 & 3,725 & 5,617 & 6,003 & 4,580 & 2,310 & 0,436 & 0,084 \\
        \bottomrule
        \multicolumn{10}{l}{$U_X = 1,7483$\,V; $U' = 1,744$\,V; $T_A = 1400$\,ns}
    \end{tabular}
\end{table}

\begin{table}[H]
    \catcode`\-=12
    \centering
    \caption{Převodník Č/A}
    \begin{tabular}{ll|ccccccc}
        \toprule
        \multirow{2}{*}{N} & [-]                        & 0      & 16     & 32     & 48     & 64      & 80      & 96      \\
                           & [-]                        & 0      & 10000  & 100000 & 110000 & 1000000 & 1010000 & 1100000 \\
        \cmidrule(rl){1-9}
        $U_{m \check{e} \check{r}}$               & [V] & -5,003 & -4,380 & -3,755 & -3,130 & -2,504  & -1,879  & -1,255  \\
        $U_a$                 & [V]                     & 0,000  & 0,314  & 0,627  & 0,941  & 1,255   & 1,569   & 1,882   \\
        $U_{NP}$                & [V]                   & -5,003 & -4,378 & -3,753 & -3,129 & -2,504  & -1,879  & -1,254  \\
        $\Delta_U$                 & [V]                & 0,000  & -0,002 & -0,002 & -0,001 & 0,000   & 0,000   & -0,001 \\
        \cmidrule(){1-9}
        \morecmidrules
        \cmidrule(){1-9}
        \multirow{2}{*}{N} & [-]                        & 112     & 128      & 144      & 160      & 176      & 192      & 208      \\
         & [-]                        & 1110000 & 10000000 & 10010000 & 10100000 & 10110000 & 11000000 & 11010000 \\
        \cmidrule(rl){1-9}
        $U_{m \check{e} \check{r}}$               & [V] & -0,630  & 0,002    & 0,623    & 1,247    & 1,872    & 2,497    & 3,121    \\
        $U_a$                 & [V]                     & 2,196   & 2,510    & 2,824    & 3,137    & 3,451    & 3,765    & 4,078    \\
        $U_{NP}$                & [V]                   & -0,630  & -0,005   & 0,620    & 1,245    & 1,869    & 2,494    & 3,119    \\
        $\Delta_U$                 & [V]                & 0,000   & 0,007    & 0,003    & 0,002    & 0,003    & 0,003    & 0,002    \\
        \cmidrule(){1-9}
        \morecmidrules
        \cmidrule(){1-5}
        \multirow{2}{*}{N} & [-]                        & 224      & 240      & 255      \\
                   & [-]                        & 11100000 & 11110000 & 11111111 \\
        \cmidrule(rl){1-5}
        $U_{m \check{e} \check{r}}$               & [V] & 3,745    & 4,370    & 4,954    \\
        $U_a$                 & [V]                     & 4,392    & 4,706    & 5,000    \\
        $U_{NP}$                & [V]                   & 3,744    & 4,368    & 4,954    \\
        $\Delta_U$                 & [V]                & 0,001    & 0,002    & 0,000    \\
        \cmidrule[0.8pt]{1-5}
        \multicolumn{9}{l}{$U_{ref} = 5$\,V}
    \end{tabular}
\end{table}

\subsection{Grafy}

\subsection{Příklady výpočtu}

\section{Seznam použitých přístrojů}

\section{Závěr}

\end{document}