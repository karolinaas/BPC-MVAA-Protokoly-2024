\documentclass[a4paper, czech]{article}

\title{Úloha č.1: Stanovení nejistot při přímém měření výkonu}
\author{Karolína Andrea Šebestová \and Filip Kokavec}
\date{Datum měření: 2.10.2024}

\usepackage[czech]{babel}
\usepackage{indentfirst}
\usepackage{graphicx}
\usepackage{float}
\usepackage[margin=1.5cm]{geometry}
\usepackage{booktabs}
\usepackage{amsmath}
\usepackage[dvipsnames]{xcolor}
\usepackage{multirow}
\usepackage{tabularray}
\usepackage{bold-extra}
\usepackage{circuitikz}
\usepackage{caption}
\usepackage{subcaption}
\usepackage[utf8]{inputenc}
\usepackage{array}

\begin{document}
\begin{table}[H]
    \centering
    \begin{tblr}{
        cell{1}{1} = {c = 2, r = 4}{c}, % Logo
        cell{1}{4} = {c = 3}{c}, % Předmět
        cell{2}{4} = {c = 3}{c}, % Jméno
        cell{3}{4} = {}{c}, % Ročník
        cell{3}{6} = {}{c}, % Studijní skupina
        cell{4}{4} = {}{c}, % Spolupracoval
        cell{4}{6} = {}{c}, % Mereno dne
        cell{5}{1} = {c = 2}{55mm}, % Kontroloval
        cell{5}{3} = {c = 2}{55mm}, % Hodnoceni
        cell{5}{5} = {c = 2}{55mm}, % Dne
        cell{6}{2} = {c = 5}{}, % Nazev ulohy
        cell{7}{1} = {}{c}, % Číslo úlohy
        cell{7}{2} = {c = 5}{c}, % Název úlohy
        vline{1,2,7} = {1.2pt},
        vline{3,5},
        hline{1,5,6,8} = {1.2pt},
        hline{2,3,4}
        }
        \includegraphics{logo_fekt.png} & & \textsuperscript{Předmět} & \large \textbf{Měření v audiotechnice} \\
             & & \textsuperscript{Jméno} & \large \textbf{Karolína Šebestová} \\
             & & \textsuperscript{Ročník} & \large \textbf{3.} & \textsuperscript{Studijní skupina} & \large \textbf{St 14:00} \\
             & & \textsuperscript{Spolupracoval} & \large \textbf{Filip Kokavec} & \textsuperscript{Měřeno dne} & \large \textbf{20.11.2024} \\
        \textsuperscript{Kontroloval} & & \textsuperscript{Hodnocení} & & \textsuperscript{Dne} \\
        \textsuperscript{Číslo úlohy} & \textsuperscript{Název úlohy} \\
        \Large \textbf{8} & \Large \textsc{\textbf{Práce s čítačem}} \\
    \end{tblr}
\end{table}

\section{Zadání}

\begin{itemize}
    \item Prostudujte blokové schéma univerzálního čítače a seznamte se s jeho principem.
    \item Čítačem změřte parametry výstupních napětí předloženého přípravku:
    \begin{itemize}
        \item kmitočet a periodu,
        \item časové parametry obou napěťových signálů,
    \end{itemize}
    \item Posuďte přesnost provedeného měření a faktory, které tuto přesnost ovlivňují.
\end{itemize}

\section{Teoretický úvod}

\section{Výsledky měření}

\subsection{Tabulky}

\begin{table}[H]
    \catcode`\-=12
    \centering
    \caption{Hodnoty naměřené čítačem HM8123}
    \begin{tabular}{ll|ccccccc}
        \toprule
        \multicolumn{2}{c}{Poznámka}          & \multicolumn{7}{c}{$\delta_{f_0} = 5 \cdot 10^{-5}\,\%$;  $f_0 = 400\,\text{MHz}$}                                                                                                    \\
        \cmidrule(rl){1-9}
        \multicolumn{2}{c}{\multirow{3}{*}{Funkce čítače}}     & FREQ A             & PER A              & \multicolumn{5}{c}{TI A$\rightarrow$B}                                                                                  \\
        \cmidrule(rl){3-3}
        \cmidrule(rl){4-4}
        \cmidrule(rl){5-9}
        \multicolumn{2}{c}{\multirow{2}{*}{}} & \multirow{2}{*}{$f$} & \multirow{2}{*}{$T$} & \multirow{2}{*}{$T_1$} & \multirow{2}{*}{$T_2$} & \multirow{2}{*}{$T_3$} & \multirow{2}{*}{$T_4$} & \multirow{2}{*}{$T_5$} \\
        \multicolumn{2}{c}{}                  &                    &                    &                     &                     &                     &                     &                     \\
        \cmidrule(rl){1-9}
        min              & [kHz], [$\mu$s]            & 3,994              & 250,4              & 12,56               & 11,08               & 15,74               & 10,36               & 37,69               \\
        max              & [kHz], [$\mu$s]            & 3,543              & 282,2              & 20,49               & 15,03               & 23,68               & 13,31               & 53,63               \\
        $\Delta_\text{ind}$             & [mHz], [ns]            & 0,1                & 1                  & 1                   & 1                   & 1                   & 1                   & 1                   \\
        $u_\text{C}$               & [Hz], [ns]             & 1,154E-3           & 0,582              & 0,577               & 0,577               & 0,577               & 0,577               & 0,577               \\
        $\tilde{U}$                & [\%]                 & 57,80E-6          & 464,8E-6           & 9,197E-3            & 10,43E-3            & 7,336E-3            & 11,14E-3            & 3,064E-3            \\
        \bottomrule
    \end{tabular}
\end{table}

\begin{table}[H]
    \catcode`\-=12
    \centering
    \caption{Hodnoty naměřené čítačem HM8123 při průměrování intervalu $T_1$}
    \begin{tabular}{ll|ccc}
        \toprule
        \multicolumn{2}{c}{\multirow{3}{*}{Funkce čítače}}     & TI A$\rightarrow$B              & \multicolumn{2}{c}{TI avg A$\rightarrow$B}   \\
        \cmidrule(rl){3-3}
        \cmidrule(rl){4-5}
        \multicolumn{2}{c}{\multirow{2}{*}{}} & \multirow{2}{*}{$T_1$} & \multicolumn{2}{c}{$T_1$ Gate time} \\
        \cmidrule(rl){4-5}
        \multicolumn{2}{c}{}                  &                     & 25\,ms           & 1\,s            \\
        \cmidrule(rl){1-5}
        min                & [$\mu$s]               & 12,56               & 12,56           & 12,56          \\
        max                & [$\mu$s]               & 20,49               & 20,49           & 20,49          \\
        $\Delta_\text{ind}$               & [ns]               & 1                   & 0,1             & 0,01           \\
        $n$                  & [-]                & -                   & 1991            & 79646          \\
        $u_\text{C}$                 & [ns]               & 577,4E-3            & 3,625E-3        & 3,624E-3       \\
        $\tilde{U}$                  & [\%]               & 9,196E-3            & 57,74E-6        & 57,74E-6       \\
        \bottomrule
    \end{tabular}
\end{table}

\subsection{Příklady výpočtu}

\begin{enumerate}
    \item Standardní nejistota měření kmitočtu
    \begin{multline*}
        u_{\text{C}_f} = \textcolor{teal}{\sqrt{\left(\frac{\delta_{f_0} \cdot f}{100\,\% \cdot \chi}\right)^2 + \left(\frac{\Delta_{f_\text{ind}}}{\chi}\right)^2}} = \sqrt{\left(\frac{5\cdot10^{-5}\,\% \cdot 3,994\cdot10^{3}\,\text{Hz}}{100\,\% \cdot \sqrt{3}}\right)^2 + \left(\frac{0,1\cdot10^{-3}\,\text{Hz}}{\sqrt{3}}\right)^2} = \hfill \\
        \ \ \ = 1,154\cdot10^{-3}\,\text{Hz} = \underline{\underline{1,154\,\text{mHz}}} \hfill
    \end{multline*}
    \item Standardní nejistota měření času
    \begin{multline*}
        u_{\text{C}_T} = \textcolor{teal}{\sqrt{\left(\frac{\delta_{f_0} \cdot T}{100\,\% \cdot \chi}\right)^2 + \left(\frac{\Delta_{T_\text{ind}}}{n \cdot \chi}\right)^2}} = \sqrt{\left(\frac{5\cdot10^{-5}\,\% \cdot 250,4\cdot10^{-6}\,\text{s}}{100\,\% \cdot \sqrt{3}}\right)^2 + \left(\frac{1\cdot10^{-9}\,\text{s}}{1 \cdot \sqrt{3}}\right)^2} = \hfill \\
        \ \ \ = 0,582\cdot10^{-9}\,\text{s} = \underline{\underline{0,582\,\text{ns}}} \hfill
    \end{multline*}
    \item Relativní rozšířená nejistota měření
    \begin{multline*}
        \tilde{U}_f = \textcolor{teal}{\frac{k_\text{r} \cdot u_\text{C}}{Y_\text{kor}} \cdot 100\,\%} = \frac{k_\text{r} \cdot u_{\text{C}_f}}{f} \cdot 100\,\% = \frac{2 \cdot 1,154\cdot10^{-3}\,\text{Hz}}{3,994\cdot10^{3}\,\text{Hz}} \cdot 100\,\% = \underline{\underline{57,78\cdot10^{-6}\,\%}} \hfill \\
        \ \ \ \tilde{U}_T = \frac{k_\text{r} \cdot u_{\text{C}_T}}{T} \cdot 100\,\% = \frac{2 \cdot 0,582\cdot10^{-9}\,\text{s}}{250,4\cdot10^{-6}\,\text{s}} \cdot 100\,\% = \underline{\underline{464,8\cdot10^{-6}\,\%}} \hfill
    \end{multline*}
    \item Počet měřených intervalů
    \begin{multline*}
        n = \frac{\text{Gate Time}}{T} = \frac{25\cdot10^{-3}\,\text{s}}{12,56\cdot10^{-6}\,\text{s}} = \underline{\underline{1990}} \hfill
    \end{multline*}
\end{enumerate}

\section{Seznam použitých přístrojů}

\section{Závěr}

\end{document}