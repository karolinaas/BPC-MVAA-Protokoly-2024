\documentclass[a4paper, czech]{article}

\title{Úloha č.1: Stanovení nejistot při přímém měření výkonu}
\author{Karolína Andrea Šebestová \and Filip Kokavec}
\date{Datum měření: 2.10.2024}

\usepackage[czech]{babel}
\usepackage{indentfirst}
\usepackage{graphicx}
\usepackage{float}
\usepackage[margin=1.5cm]{geometry}
\usepackage{booktabs}
\usepackage{amsmath}
\usepackage[dvipsnames]{xcolor}
\usepackage{multirow}
\usepackage{tabularray}
\usepackage{bold-extra}
\usepackage{circuitikz}
\usepackage{caption}
\usepackage{subcaption}
\usepackage[utf8]{inputenc}
\usepackage{array}

\begin{document}
\begin{table}[H]
    \centering
    \begin{tblr}{
        cell{1}{1} = {c = 2, r = 4}{c}, % Logo
        cell{1}{4} = {c = 3}{c}, % Předmět
        cell{2}{4} = {c = 3}{c}, % Jméno
        cell{3}{4} = {}{c}, % Ročník
        cell{3}{6} = {}{c}, % Studijní skupina
        cell{4}{4} = {}{c}, % Spolupracoval
        cell{4}{6} = {}{c}, % Mereno dne
        cell{5}{1} = {c = 2}{55mm}, % Kontroloval
        cell{5}{3} = {c = 2}{55mm}, % Hodnoceni
        cell{5}{5} = {c = 2}{55mm}, % Dne
        cell{6}{2} = {c = 5}{}, % Nazev ulohy
        cell{7}{1} = {}{c}, % Číslo úlohy
        cell{7}{2} = {c = 5}{c}, % Název úlohy
        vline{1,2,7} = {1.2pt},
        vline{3,5},
        hline{1,5,6,8} = {1.2pt},
        hline{2,3,4}
        }
        \includegraphics{logo_fekt.png} & & \textsuperscript{Předmět} & \large \textbf{Měření v audiotechnice} \\
             & & \textsuperscript{Jméno} & \large \textbf{Karolína Šebestová} \\
             & & \textsuperscript{Ročník} & \large \textbf{3.} & \textsuperscript{Studijní skupina} & \large \textbf{St 14:00} \\
             & & \textsuperscript{Spolupracoval} & \large \textbf{Filip Kokavec} & \textsuperscript{Měřeno dne} & \large \textbf{20.11.2024} \\
        \textsuperscript{Kontroloval} & & \textsuperscript{Hodnocení} & & \textsuperscript{Dne} \\
        \textsuperscript{Číslo úlohy} & \textsuperscript{Název úlohy} \\
        \Large \textbf{8} & \Large \textsc{\textbf{Práce s čítačem}} \\
    \end{tblr}
\end{table}

\section{Zadání}

\begin{itemize}
    \item Prostudujte blokové schéma univerzálního čítače a seznamte se s jeho principem.
    \item Čítačem změřte parametry výstupních napětí předloženého přípravku:
    \begin{itemize}
        \item kmitočet a periodu,
        \item časové parametry obou napěťových signálů,
    \end{itemize}
    \item Posuďte přesnost provedeného měření a faktory, které tuto přesnost ovlivňují.
\end{itemize}

\section{Teoretický úvod}

Pro měření veličin kmitočtu, délky periody nebo časového intervalu je vhodné použít elektronické zařízení univerzální čítač.
Jeho další možné funkce jsou měření průměrné délky periody, poměr dvou kmitočtů, šířky impulzu, doby náběžné nebo sestupné hrany, činitel plnění impulzu a nebo fázový posun.

Čítač pracuje na principu měření doby uplynuté mezi dvěma změnami.
Čítač je schopen rozeznat změny v amplitudě a také je schopen rozeznat zda je změna stoupající nebo klesající.
Taktéž dokáže porovnávat změny ve více signálech.

Úroveň, ve které čítač zaznamenává změnu se odborně nazývá komparační úroveň.
Měření reálných signálů se vyznačuje také konečně strmými hranami změn úrovní.
Abychom předešli možné odchylce v měření, nastavujeme komparační úroveň na 50\,\%
maximální úrovně impulsu.

\begin{figure}[H]
    \centering
    \begin{circuitikz}[european]
        \tikzset{flipflop RS/.style={flipflop, flipflop def={t1=R, t3=S, t6=Q, t4={\ctikztextnot{Q}}}}}
        \ctikzset{logic ports=european, tripoles/european not symbol=ieee circle}

        \draw (0,0) node[flipflop RS, dot on notQ](RSFF){}
        (RSFF.pin 1) node[european and port, anchor=out](ANDR){}
        (RSFF.pin 3) node[european and port, anchor=out](ANDS){}
        (ANDS.in 2) -- +(0,-0.75) coordinate(X) (X) -- (X -| RSFF.pin 4) -- (RSFF.pin 4)
        (ANDR.in 1) -- +(0,0.75) coordinate(X) (X) -- node[above]{Klopný obvod} (X -| RSFF.pin 6) -- (RSFF.pin 6) node[circ](RSout){}
        (RSout) -- +(0.5,0) node[european and port, anchor={in 1}](ANDout){}
        (ANDout.south) node[below]{Hradlo}
        (ANDout.in 2) -- ++(0,-2) -- ++(0.5,0) 
        node[draw, rectangle, anchor=west, align=center, line width=0.8, minimum height=1cm, inner xsep=7.5]{Kmitočtový\\normál $f_0$}
        (ANDout.out) +(-0.4,0) edge[-{Stealth[scale=1.5]}, thick] +(1,0) ++(1,0)
        node[draw, rectangle, anchor=west, align=center, line width=0.8, minimum height=1cm, inner xsep=7.5](DesČ){Desítkový\\čítač}
        (DesČ.south) edge[{Stealth[scale=1.5]}-, thick] ++(0,-1) ++(0,-1) node[anchor=north, align=center]{Nulování\\čítače}
        (DesČ.north) edge[-{Stealth[scale=1.5]}, thick] ++(0,1) ++(0,1) node[draw, rectangle, anchor=south, align=center, line width=0.8, minimum height=1cm, inner xsep=7.5](paměť){Paměť a\\zobrazení}
        (paměť.west) edge[{Stealth[scale=1.5]}-, thick] ++(-1,0) ++(-1,0) node[anchor=east, align=center]{Zápis do\\paměti}
        (ANDR.in 2) node[cute spdt up, xscale=-1, anchor=in](sw){} (sw.east) +(-0.4,0) node[left,align=center]{Přepínač\\funkce}
        (sw.out 1) node[left]{2} (sw.out 2) node[left]{1}

        (sw-out 1.n) -- ++(0,2.5) -- ++(-1,0) node[circ](uzel){} to[R] ++(0,-2) node[tground]{}
        (uzel) to[C] ++(-1.5,0) node[draw, rectangle, anchor=east, align=center, line width=0.8, minimum height=1cm, inner xsep=7.5](komp){Komparátor\\B}
        (komp.west) ++(0,0.25) -- ++(-1,0) node[ocirc, scale=2, label=above:{VSTUP B}]{}
        (komp.west) ++(0,-0.25) -- ++(-0.45,0) -- ++(0,-1.2) node[genericpotentiometershape, anchor=wiper, rotate=270](pot){}
        (pot.north) ++(0.2,0) node[right, align=center]{Nastavení\\komparační\\úrovně}
        (pot.west) -- +(0,0.4) node[left]{$+$}
        (pot.east) -- +(0,-0.4) node[left]{$-$}
        (uzel) +(0,0.4) node[above]{Tvarovací obvod}

        (ANDS.in 1) -- (ANDS.in 1 -| sw-out 2.s) node[circ](uzel){} -- (sw-out 2.s)
        (uzel) -- ++(0,-1) -- ++(-1,0) node[circ](uzel){} to[R] ++(0,-2) node[tground]{} node[below]{Tvarovací obvod}
        (uzel) to[C] ++(-1.5,0) node[draw, rectangle, anchor=east, align=center, line width=0.8, minimum height=1cm, inner xsep=7.5](komp){Komparátor\\A}
        (komp.west) ++(0,0.25) -- ++(-1,0) node[ocirc, scale=2, label=above:{VSTUP A}]{}
        (komp.west) ++(0,-0.25) -- ++(-0.45,0) -- ++(0,-1.2) node[genericpotentiometershape, anchor=wiper, rotate=270](pot){}
        (pot.north) ++(0.2,0) node[right, align=center]{Nastavení\\komparační\\úrovně}
        (pot.west) -- +(0,0.4) node[left]{$+$}
        (pot.east) -- +(0,-0.4) node[left]{$-$};
    \end{circuitikz}
    \caption{Blokové schéma univerzálního čítače}
\end{figure}

\section{Výsledky měření}

\subsection{Tabulky}

\begin{table}[H]
    \catcode`\-=12
    \centering
    \caption{Hodnoty naměřené čítačem HM8123}
    \begin{tabular}{ll|ccccccc}
        \toprule
        \multicolumn{2}{c}{Poznámka}          & \multicolumn{7}{c}{$\delta_{f_0} = 5 \cdot 10^{-5}\,\%$;  $f_0 = 400\,\text{MHz}$}                                                                                                    \\
        \cmidrule(rl){1-9}
        \multicolumn{2}{c}{\multirow{3}{*}{Funkce čítače}}     & FREQ A             & PER A              & \multicolumn{5}{c}{TI A$\rightarrow$B}                                                                                  \\
        \cmidrule(rl){3-3}
        \cmidrule(rl){4-4}
        \cmidrule(rl){5-9}
        \multicolumn{2}{c}{\multirow{2}{*}{}} & \multirow{2}{*}{$f$} & \multirow{2}{*}{$T$} & \multirow{2}{*}{$T_1$} & \multirow{2}{*}{$T_2$} & \multirow{2}{*}{$T_3$} & \multirow{2}{*}{$T_4$} & \multirow{2}{*}{$T_5$} \\
        \multicolumn{2}{c}{}                  &                    &                    &                     &                     &                     &                     &                     \\
        \cmidrule(rl){1-9}
        min              & [kHz], [$\mu$s]            & 3,994              & 250,4              & 12,56               & 11,08               & 15,74               & 10,36               & 37,69               \\
        max              & [kHz], [$\mu$s]            & 3,543              & 282,2              & 20,49               & 15,03               & 23,68               & 13,31               & 53,63               \\
        $\Delta_\text{ind}$             & [mHz], [ns]            & 0,1                & 1                  & 1                   & 1                   & 1                   & 1                   & 1                   \\
        $u_\text{C}$               & [Hz], [ns]             & 1,154E-3           & 0,582              & 0,577               & 0,577               & 0,577               & 0,577               & 0,577               \\
        $\tilde{U}$                & [\%]                 & 57,80E-6          & 464,8E-6           & 9,197E-3            & 10,43E-3            & 7,336E-3            & 11,14E-3            & 3,064E-3            \\
        \bottomrule
    \end{tabular}
\end{table}

\begin{table}[H]
    \catcode`\-=12
    \centering
    \caption{Hodnoty naměřené čítačem HM8123 při průměrování intervalu $T_1$}
    \begin{tabular}{ll|ccc}
        \toprule
        \multicolumn{2}{c}{\multirow{3}{*}{Funkce čítače}}     & TI A$\rightarrow$B              & \multicolumn{2}{c}{TI avg A$\rightarrow$B}   \\
        \cmidrule(rl){3-3}
        \cmidrule(rl){4-5}
        \multicolumn{2}{c}{\multirow{2}{*}{}} & \multirow{2}{*}{$T_1$} & \multicolumn{2}{c}{$T_1$ Gate time} \\
        \cmidrule(rl){4-5}
        \multicolumn{2}{c}{}                  &                     & 25\,ms           & 1\,s            \\
        \cmidrule(rl){1-5}
        min                & [$\mu$s]               & 12,56               & 12,56           & 12,56          \\
        max                & [$\mu$s]               & 20,49               & 20,49           & 20,49          \\
        $\Delta_\text{ind}$               & [ns]               & 1                   & 0,1             & 0,01           \\
        $n$                  & [-]                & -                   & 1991            & 79646          \\
        $u_\text{C}$                 & [ns]               & 577,4E-3            & 3,625E-3        & 3,624E-3       \\
        $\tilde{U}$                  & [\%]               & 9,196E-3            & 57,74E-6        & 57,74E-6       \\
        \bottomrule
    \end{tabular}
\end{table}

\subsection{Příklady výpočtu}

\begin{enumerate}
    \item Standardní nejistota měření kmitočtu
    \begin{multline*}
        u_{\text{C}_f} = \textcolor{teal}{\sqrt{\left(\frac{\delta_{f_0} \cdot f}{100\,\% \cdot \chi}\right)^2 + \left(\frac{\Delta_{f_\text{ind}}}{\chi}\right)^2}} = \sqrt{\left(\frac{5\cdot10^{-5}\,\% \cdot 3,994\cdot10^{3}\,\text{Hz}}{100\,\% \cdot \sqrt{3}}\right)^2 + \left(\frac{0,1\cdot10^{-3}\,\text{Hz}}{\sqrt{3}}\right)^2} = \hfill \\
        \ \ \ = 1,154\cdot10^{-3}\,\text{Hz} = \underline{\underline{1,154\,\text{mHz}}} \hfill
    \end{multline*}
    \item Standardní nejistota měření času
    \begin{multline*}
        u_{\text{C}_T} = \textcolor{teal}{\sqrt{\left(\frac{\delta_{f_0} \cdot T}{100\,\% \cdot \chi}\right)^2 + \left(\frac{\Delta_{T_\text{ind}}}{n \cdot \chi}\right)^2}} = \sqrt{\left(\frac{5\cdot10^{-5}\,\% \cdot 250,4\cdot10^{-6}\,\text{s}}{100\,\% \cdot \sqrt{3}}\right)^2 + \left(\frac{1\cdot10^{-9}\,\text{s}}{1 \cdot \sqrt{3}}\right)^2} = \hfill \\
        \ \ \ = 0,582\cdot10^{-9}\,\text{s} = \underline{\underline{0,582\,\text{ns}}} \hfill
    \end{multline*}
    \item Relativní rozšířená nejistota měření
    \begin{multline*}
        \tilde{U}_f = \textcolor{teal}{\frac{k_\text{r} \cdot u_\text{C}}{Y_\text{kor}} \cdot 100\,\%} = \frac{k_\text{r} \cdot u_{\text{C}_f}}{f} \cdot 100\,\% = \frac{2 \cdot 1,154\cdot10^{-3}\,\text{Hz}}{3,994\cdot10^{3}\,\text{Hz}} \cdot 100\,\% = \underline{\underline{57,78\cdot10^{-6}\,\%}} \hfill \\
        \ \ \ \tilde{U}_T = \frac{k_\text{r} \cdot u_{\text{C}_T}}{T} \cdot 100\,\% = \frac{2 \cdot 0,582\cdot10^{-9}\,\text{s}}{250,4\cdot10^{-6}\,\text{s}} \cdot 100\,\% = \underline{\underline{464,8\cdot10^{-6}\,\%}} \hfill
    \end{multline*}
    \item Počet měřených intervalů
    \begin{multline*}
        n = \frac{\text{Gate Time}}{T} = \frac{25\cdot10^{-3}\,\text{s}}{12,56\cdot10^{-6}\,\text{s}} = \underline{\underline{1990}} \hfill
    \end{multline*}
\end{enumerate}

\section{Seznam použitých přístrojů}

\begin{itemize}
    \item Univerzální čítač Rohde \& Schwarz HM8123. v.č. 100042237463
    \item Digitální osciloskop Rigol MSO1074, v.č. DS9ZD201400322
    \item Školní stabilizovaný zdroj Tesla BK 125, v.č. 904417
\end{itemize}

\section{Závěr}

Největší výhodoz univerzálního čítače je jeho rychlost vzorkování a schopnost nastavení náběžné
nebo sestupné hrany.
Oproti digitálnímu osciloskopu je jeho zřejmou nevýhodou jeho úzké zaměření oproti relativní
všestrannosti osciloskopů, jeho výhodou je však velmi vysoká přednost, kdy naše
výsledky vyšly s odchylkou v řádech $10^{-3}$ a nižší.

Čítač je také schopen měřit signály o vyšších kmitočtech než obdobný osciloskop,
který nedosahuje tak vysokých vzorkovacích kmitočtů.
Osciloskop je také schopen graficky zobrazit měřený signál a tedy je možné udělat mnohem
lepší představu o jeho průběhu.

Zvýšení nastavené hodnoty Gate Time na čítači výrazně zvýšilo přesnost provedeného měření.
Další navyšování hodnoty Gate Time už však k výraznému zvýšení přesnosti nevedlo.

\end{document}